\documentclass[11pt, oneside]{amsart}   	% use "amsart" instead of "article" for AMSLaTeX format
\usepackage{geometry}                		% See geometry.pdf to learn the layout options. There are lots.
\geometry{letterpaper}                   		% ... or a4paper or a5paper or ... 
%\geometry{landscape}                		% Activate for for rotated page geometry
%\usepackage[parfill]{parskip}    		% Activate to begin paragraphs with an empty line rather than an indent
\usepackage{graphicx}				% Use pdf, png, jpg, or eps with pdflatex; use eps in DVI mode
								% TeX will automatically convert eps --> pdf in pdflatex		
\usepackage{amssymb}
\usepackage{amsmath}
\usepackage{subfig}


\title{COVID-19 Individual Based Model  with Instantaneous Contract Tracing}
\author{Rob Hinch, Will Probert, Anel Nurtay}
%\date{}							% Activate to display a given date or no date

\begin{document}
\maketitle

\section{Overview}
The individual based model (IBM) is for simulating the spread of COVID-19 in a city and to analyse the effect of both passive and active intervention strategies.
The model includes demographic data, which control both the dynamics of the interactions of individuals as well as the the outcome of the disease.
The disease is spread via interaction between individuals which are remembered to facilitate contact tracing.
Intervention strategies such as self-quarantining, testing and contact-tracing can then be analysed.

\section{Demographics}

The demographics of the model are based upon UK-wide data for 2018 from the Office of National Statistics. 
Individuals are put in one of 3 categories: child (0-17 years), adult (18-64 years) and elderly (65+).

\medskip \medskip \medskip
\begin{tabular}{ |p{4cm}||p{7cm}|p{1cm}|  }
 \hline
 \multicolumn{3}{|c|}{Demographic Parameters} \\
 \hline
 Name   & Description & Value \\
 \hline
 \hline 
uk_pop_0_17    & UK population 0_17 years old     & 14.05m \\
uk_pop_18_64  & UK population 18-64 years old   & 40.22m \\
uk_pop_65        & UK population 65+ years old      & 10.04m \\
 \hline
\end{tabular}


\section{Interaction Network}

\section{Disease Dynamics}

\section{Passive Interventions}

\section{Active Interventions}

\section{References}


\end{document}  
