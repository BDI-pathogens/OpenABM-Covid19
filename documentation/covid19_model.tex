\documentclass[11pt, oneside]{amsart}   	% use "amsart" instead of "article" for AMSLaTeX format
\usepackage{geometry}                		% See geometry.pdf to learn the layout options. There are lots.
\geometry{letterpaper}                   		% ... or a4paper or a5paper or ... 
%\geometry{landscape}                		% Activate for for rotated page geometry
%\usepackage[parfill]{parskip}    		% Activate to begin paragraphs with an empty line rather than an indent
\usepackage{graphicx}				% Use pdf, png, jpg, or eps with pdflatex; use eps in DVI mode
								% TeX will automatically convert eps --> pdf in pdflatex		
\usepackage{amssymb}
\usepackage{amsmath}
\usepackage{subfig}
\newcommand{\us}{\textunderscore}


\title{COVID-19 Individual Based Model  with Instantaneous Contract Tracing}
\author{Rob Hinch, Will Probert, Anel Nurtay, Christophe Fraser}
%\date{}							% Activate to display a given date or no date

\begin{document}
\maketitle

\section{Overview}
The individual based model (IBM) is for simulating the spread of COVID-19 in a city and to analyse the effect of both passive and active intervention strategies.
The model includes demographic data, which control both the dynamics of the interactions of individuals as well as the the outcome of the disease.
The disease is spread via interaction between individuals which are remembered to facilitate contact tracing.
Intervention strategies such as self-quarantining, testing and contact-tracing can then be analysed.

\section{Demographics}

The demographics of the model are based upon UK-wide data for 2018 from the Office of National Statistics (ONS). 
Individuals are put in one of 3 categories: child (0-17 years), adult (18-64 years) and elderly (65+).
Every individual is part of household which forms an important part of each persons daily interactions.
We use household size data from the ONS.

\medskip \medskip
\begin{tabular}{ |p{4cm}|p{8.5cm}|p{1.5cm}|  }
 \hline
 \multicolumn{3}{|c|}{Demographic Parameters} \\
 \hline
 Name   & Description & Value \\
 \hline
 \hline 
uk\us pop\us 0\us 17    & UK population 0-17 years old  (millions)  & 14.05 \\
uk\us pop\us 18\us 64  & UK population 18-64 years old  (millions)  & 40.22 \\
uk\us pop\us 65        & UK population 65+ years old (millions)       & 10.04 \\
 \hline 
uk\us house\us1 & UK households with 1 person (thousands) & 8,198 \\
uk\us house\us2 & UK households with 2 person (thousands) & 9,609 \\
uk\us house\us3 & UK households with 3 person (thousands) & 4,287 \\
uk\us house\us4 & UK households with 4 person (thousands) & 3,881 \\
uk\us house\us5 & UK households with 5 person (thousands) & 1,254 \\
uk\us house\us6 & UK households with 6 person (thousands) & 596 \\
 \hline
\end{tabular}
\medskip \medskip

\section{Interaction Network}

Interactions between individuals are modelled via membership of numerous networks which represent peoples daily interactions.
The membership of different networks leads to age-group assortativity in the interactions.
Previous studies of social contacts for infectious disease modelling has estimated the mean number of interactions that individuals have by age group (Mossong, 2008).
We estimate mean interactions by age-group by aggregating this data 

\medskip \medskip
\begin{table}
\centering
\begin{tabular}{ |p{5cm}|p{1.5cm}|  }
 \hline
 \multicolumn{2}{|c|}{Mean daily interactions} \\
 \hline
Age-Group  & Value \\
 \hline
 \hline 
children (0-17 years) & 15 \\
adults (18-64 years) & 13 \\
elderly (65 years+) & 7 \\
 \hline
\end{tabular}
\end{table}
\medskip \medskip

Our model contains 3 types of networks. One to represent households, one work-place (for children this would be school) and one random daily interactions.

\subsection{Household Network}
Each individual is assigned to live in a household, with the proportion of people in each household taken from the UK household data (see demographics section).
Each day, each person has an interaction with everybody within their household.
Elderly people are assumed to live in either 1 or 2 person household with other elderly people, with the ratio of elderly 1 and 2 person households being the same as the general population.
Children are assumed to live in household with two adults (so they can only live in 3/4/5/6 person households). 
The proportion of 3/4/5/6 person with children is the same as the those with adults only.

\subsection{Work-place Network}
Each individual is part of a single work-place network.
The work-place networks are Watts-Strogatz small-world networks.
There is one network for each age group, with the child and elderly network containing a small proportion of adults (i.e.\ teachers and carers).
When constructing the work-place networks we randomly sort the individuals, so there is no link between the household interactions and the local interactions on the small-world network.
Every day each person interacts with a random subset of their connections on their work-place network.

\medskip \medskip
\begin{tabular}{ |p{5cm}|p{7.5cm}|p{1.5cm}|  }
 \hline
 \multicolumn{3}{|c|}{Work-place Network Parameters} \\
 \hline
 Name   & Description & Value \\
 \hline
 \hline 
mean\us work\us interaction\us child    & mean number of connections for children & 10 \\
mean\us work\us interaction\us adult   & mean number of connections for adults & 7 \\
mean\us work\us interaction\us elderly & mean number of connections for elderly & 3 \\
\hline
child\us network\us adults & fraction of adults in child network & 0.2 \\
elderly\us network\us adults & fraction of adults in elderly network & 0.2 \\
\hline
daily\us fraction\us work & fraction of daily work connections made & 0.5 \\
prob\us network\us rewire (*) & probability of rewiring a connection in the Watts-Stogatz small-world network & 0.1 \\ 
 \hline
\end{tabular}
\medskip \medskip

The difference in the number of interactions for each age group is due to the overall number of daily interactions that each group have

\subsection{Random Network}
In addition to the recurring structured networks of households and work-places, we include a number of random interactions as well.
These interactions are drawn each day and are independent of the previous days connections.
The number of random connections an individual makes is the same each day (without interventions) and is drawn at the start of the simulation from a negative-binomial distribution.
This variation in the number of interactions introduces "super-spreaders" in to the network who make much larger numbers of interactions than the average.

\medskip \medskip
\begin{tabular}{ |p{5.6cm}|p{7cm}|p{1.5cm}|  }
 \hline
 \multicolumn{3}{|c|}{Random Network Parameters} \\
 \hline
 Name   & Description & Value \\
 \hline
 \hline 
mean\us random\us interaction\us child    & mean number of connections for children & 2 \\
mean\us random\us interaction\us adult    & mean number of connections for adults & 4 \\
mean\us random\us interaction\us elderly & mean number of connections for elderly & 3 \\
\hline
sd\us random\us interaction\us child   (*)  & s.d.\ number of connections for children & 2 \\
sd\us random\us interaction\us adult   (*)  & s.d.\ number of connections for adults & 4 \\
sd\us random\us interaction\us elderly (*) & s.d.\ number of connections for elderly & 3 \\
 \hline
\end{tabular}
\medskip \medskip

The mean numbers of connections were chosen so that the total number of daily interactions matched that from the social interaction studies. 
The split between work and random interactions was chosen to be lower in children. 
Each day a list is made which contains all individuals who make random interactions and each person is repeated by the number of interactions they make.
This list is then randomly shuffled and interactions are made between adjacent pairs on the shuffled list.

\section{Infection Dynamics}

The disease is spread by interactions between infected and susceptible individuals.
The probability of transmission is determined by the the status of the infector and the age of the susceptible.
Note that the type of interactions (i.e.\ household, work or random) or the length of the interaction (currently not modelled) is used in deciding the likelihood of transmission.
From early studies of coronavirus and analyses of other epidemics, we know that immediately after become infected an individual is not infectious.
The level of infectiousness increase with time and peaks typically 5-7 days after the initial infection before decreasing.

\section{Disease Dynamics}


\section{Passive Interventions}

\section{Active Interventions}

\section{References}


\end{document}  
